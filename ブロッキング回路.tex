トランスの結合条件は
\begin{cases}
L_1=L_2=L\\
n_1:n_2=1:1\\
K=1 \qquad (結合係数)
\end{cases}
とする。
トランジスタ\(T_r\)の特性
\begin{cases}
h_{FE} &: 直流電流増幅率\\
V_{CE(SAT)} &: {\small \textrm{C-E}} 間飽和電圧\\
V_{BE} &: {\small \textrm{B-E}}間順方向電圧
\end{cases}

\begin{cases}
v_1(t)&=&V_{cc}+v_{L1}(t) \tag{*} \\
v_2(t)&=&V_{cc}-v_{L2}(t)\\
v_{L2}(t)&=&v_{L1}(t)
\end{cases}
よって
\begin{eqnarray}
v_2(t)&=&V_{cc}-v_{L1}(t) \notag \\
&=&V_{cc}-(v_1(t)-V_{cc}) \notag \\
&=&2V_{cc}-v_1(t)\\
\end{eqnarray}
\(T_r= \textrm{ON} \)  期間(\(T_1\)期間)
コイルの基本式より
$$I_C=\frac{1}{L} \int_{0}^{T_1} v_{L1}(t) dt$$
また、
$$v_{L1}(t)=V_{cc}-V_{CE(SAT)}$$
であり時間によらず一定値であるので
\begin{eqnarray}
I_C&=&\frac{1}{L} \int_{0}^{T_1} (V_{cc}-v_{CE(SAT)})dt \notag \\
&=&\frac{V_{cc}-V_{CE(SAT)}}{L}\int_{0}^{T_1} dt \notag \\
&=&\frac{V_{cc}-V_{CE(SAT)}}{L}\cdot T_1
\end{eqnarray}
となる。

次に電磁誘導(トランス結合)によりコイル\(L_2\)に起電する電圧は\(v_{L2}(t)\)は\(v_{L1}(t)\)と逆相なので、
\begin{equation}
v_{L2}(t)=-(V_{cc}-V_{CE(SAT)})
\end{equation}
\(T_r\)のベースに流れ入る電流\(I_b\)は
$$I_b=\frac{V_{cc}-v_{L2}(t)-V_{be}}{R_b}$$
4,5より
$$I_b=\frac{2V_{cc}-V_{CE(SAT)}-V_{be}}{R_b}\tag{6}\label{Ib}$$
\(T_r\)の特性より
$$I_c=h_{FE}\cdot I_b\tag{7}\label{Ic}$$
\[\because
\begin{cases}
T_rがON時であり\\
V_{CE(SAT)}\ll LEDの順方向電圧なので\\
R_Cに流れる電流は無視できる。
\end{cases}\]
\eqref{Ib},\eqref{Ic}より
$$I_c=h_{FE}\cdot \frac{2V_{cc}-V_{CE(SAT)}-V_{be}}{R_b}$$
また、
\begin{cases}
V_{CE(SAT)}=0\\
V_{BE}=0
\end{cases}
とした場合は
$$I_c=h_{FE}\cdot \frac{2V_{CC}}{R_b}$$
となる。
次に\(T_1\)を求める。
$$T_1=h_{FE}\cdot\frac{L}{R_b} \cdot \frac{2V_{cc}-V_{CE(SAT)}-V_{BE}}{V_CC-V_{CE(SAT)}}$$
また、
\begin{cases}
V_{CE(SAT)}=0\\
V_{BE}=0
\end{cases}
とした場合は、
$$T_1=h_{FE}\cdot \frac{2L}{R_b}$$
となる。
このようにコイルに蓄積出来る有限電流値\(I_C\)および有限時間\(T_1\)が生成される。